\documentclass[a4paper,12pt]{article}
\usepackage{graphicx}
\usepackage{listings}
\usepackage{amsmath}
\usepackage{color}
\usepackage{titlesec}


\begin{document}
\title{Report Summary}
\author{Martin Moran}
\maketitle

\section{Introduction}
In this section we will introduce the concept of using HPC to study genetics. We must include the motivation of doing this work and place special importance on how HPC can be used to improve this [i.e allow us to achieve this in a reasonable timescale]

\section{Background}
We must introduce the genetic concepts used in this dissertation, such as DNA and SNPs. We must also explain EHH, IHH and IHS as well as the concepts of N-wise and conditional analysis.

\section{Design}
In this section we will discuss the overall design of the program. We must explain the key areas of the original program and discuss how these are changed to implement N-wise and conditional analysis

\section{Implementation}
This section will contain the specific tools used to implement the program. We should discuss the use of openMP and MPIRPC specifically.

\section{Results and Analysis}
This section should present results and benchmarks of the program. We will want to benchmark the program on Archer. These should be analysed with conclusions taken from  these results

\subsection{Datasets}
We must choose datasets which are useful[Redo]. There is a known pair of SNPs which are positively selected in chromosome 6 but chromosome 6 contains a section

\section{Future Work}
In this section we will set out some of the possible avenues of study to follow in future work.

\section{Conclusions}
In this section we will present some conclusions to the work done for the dissertation

\end{document}